\documentclass[10pt,fleqn]{article} % Default font size and left-justified equations
\input{style/new_style}

%\newif\ifprof
%%\proftrue
%\proffalse


\def\discipline{Sciences Industrielles de l'Ingénieur}
\def\xxtitre{%
\ifxp
CI ** : 
\else
\fi
}

\def\xxsoustitre{%
Chapitre ** -- }

\def\xxauteur{%
Xavier \textsc{Pessoles}
}

\def\xxpied{%
CI ** : **\\
Ch. ** : ** -- Cours TD
}

%---------------------------------------------------------------------------



\begin{document}

%\part{TEST PART}
%\chapterimage{chapter_head_2.pdf} % Chapter heading image
%\chapter{Text Chapter}
\pagestyle{empty}

\begin{minipage}{1cm}
\rotatebox{90}{\LARGE\sffamily\textsc{\color{ocre}\textbf{Partie 1}}}
\end{minipage} \hfill
\begin{minipage}[c]{9cm}
\begin{titrepartie}
\begin{flushright}
\renewcommand{\baselinestretch}{1.1} 
\Large\sffamily\textsc{\textbf{Étude cinématique des systèmes de solides de la chaîne d'énergie
Analyser, Modéliser, Résoudre}}
\renewcommand{\baselinestretch}{1} 
\end{flushright}
\end{titrepartie}

\end{minipage} \hfill
\begin{minipage}[c]{3.5cm}
{\large\sffamily\textsc{\textbf{\color{ocre} Sciences \\ Industrielles de \\ l'Ingénieur}}}
\end{minipage} 

\vspace{2cm}

\begin{minipage}{2cm}
\LARGE\sffamily\textsc{\color{ocre}\textbf{Cours}}
\end{minipage} \hfill
\begin{minipage}[c]{12cm}
\begin{titrechapitre}
\renewcommand{\baselinestretch}{1.1} 

\Large\sffamily\textsc{\textbf{Chapitre 1}}

\Large\sffamily\textsc{\textbf{Titre du chapitre blla bla bkabka bkabakbakb bkabakbakb bkabakbakb bkabakbakb}}

\vspace{.5cm}
\renewcommand{\baselinestretch}{1} 
\normalsize\normalfont
\textsl{\textbf{Concevoir :}
\begin{itemize}[label=\ding{112},font=\color{ocre}] 
\item Conc1-C2 : Démarche de conception appliquée aux fonctions techniques;
\item Conc1-C3.4 : Critères de choix pour la fonction : la fonction assemblage;
\item Conc2-C5 : Méthodes de conception.
\end{itemize}}

\renewcommand{\baselinestretch}{1} 

\end{titrechapitre}

\end{minipage} 

\vspace{2cm}

\begin{flushright}
\begin{minipage}{.6\linewidth}
\startcontents
\printcontents{}{1}{}
\end{minipage}
\end{flushright}

\newpage
\pagestyle{fancy}


\section{Titre 1}
\subsection{Sous titre 1.1}
\lipsum[1-2]

\begin{exercise}[toto]
This is a good place to ask a question to test learning progress or further cement ideas into students' minds.
\end{exercise}

\begin{corollary}[Corollary name]
The concepts presented here are now in conventional employment in mathematics. Vector spaces are taken over the field $\mathbb{K}=\mathbb{R}$, however, established properties are easily extended to $\mathbb{K}=\mathbb{C}$.
\end{corollary}

\begin{remark}
The concepts presented here are now in conventional employment in mathematics. Vector spaces are taken over the field $\mathbb{K}=\mathbb{R}$, however, established properties are easily extended to $\mathbb{K}=\mathbb{C}$.
\end{remark}

\begin{theorem}
The concepts presented here are now in conventional employment in mathematics. Vector spaces are taken over the field $\mathbb{K}=\mathbb{R}$, however, established properties are easily extended to $\mathbb{K}=\mathbb{C}$.
\end{theorem}

\begin{theorem}[Titre]
The concepts presented here are now in conventional employment in mathematics. Vector spaces are taken over the field $\mathbb{K}=\mathbb{R}$, however, established properties are easily extended to $\mathbb{K}=\mathbb{C}$.
\end{theorem}

\begin{definition}[Titre]
The concepts presented here are now in conventional employment in mathematics. Vector spaces are taken over the field $\mathbb{K}=\mathbb{R}$, however, established properties are easily extended to $\mathbb{K}=\mathbb{C}$.
\end{definition}

\begin{vocabulary}[Titre - Vocab]
The concepts presented here are now in conventional employment in mathematics. Vector spaces are taken over the field $\mathbb{K}=\mathbb{R}$, however, established properties are easily extended to $\mathbb{K}=\mathbb{C}$.
\end{vocabulary}

\subsection{Sous titre 1.2}
\lipsum[1-2]
\subsection{Sous titre 1.3}
\lipsum[1-2]
\section{Titre 2}
\subsection{Sous titre 2.1}
\subsection{Sous titre 2.2}
\subsection{Sous titre 2.3}
\end{document}


